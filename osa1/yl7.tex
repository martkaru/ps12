\section{Ülesanne 7}
Abstraktse masina semantika täiendused on:
\\
\(\langle \text{DUP:}c,\ z\text{:}e,\ s \rangle \triangleright \langle c,\ z\text{:}z\text{:}e,\ s \rangle \), kus \(z \in (Z \cup T) \)
\\
\(\langle \text{SWAP:}c,\ z_1\text{:}z_2\text{:}e,\ s \rangle \triangleright \langle c,\ z_2\text{:}z_1\text{:}e,\ s \rangle \), kus \(z \in (Z \cup T) \)
\\
\\
Osa 2: Näita, et 
\\
\(
  CS\llbracket x:=z-y \rrbracket : CS\llbracket w:= x * z \rrbracket
  \)
ja
\\
\texttt{\footnotesize FETCH z : DUP : FETCH y : SWAP : SUB : DUP : STORE x : MUL : STORE w}
\\
on samaväärsed.
\\
Esimesest koodilõigust saab kompileerida:
\\
\\
\(
  CS\llbracket x:=z-y \rrbracket : CS\llbracket w:= x * z \rrbracket \Rightarrow\\
  CA\llbracket z-y \rrbracket : \texttt{\footnotesize STORE x} : CA\llbracket x*z \rrbracket : \texttt{\footnotesize STORE w} \Rightarrow\\
  CA\llbracket y \rrbracket : CA\llbracket z \rrbracket : \texttt{\footnotesize SUB} : \texttt{\footnotesize STORE x} : 
    CA\llbracket z \rrbracket : CA\llbracket x \rrbracket : \texttt{\footnotesize MUL}  : \texttt{\footnotesize STORE w} \Rightarrow\\
  \texttt{\footnotesize FETCH y}  : \texttt{\footnotesize FETCH z} : \texttt{\footnotesize SUB} : \texttt{\footnotesize STORE x} : 
    \texttt{\footnotesize FETCH z} : \texttt{\footnotesize FETCH x} : \texttt{\footnotesize MUL}  : \texttt{\footnotesize STORE w} \\
\)
\\
Samaväärsuse tõestus:
\\
Esimesest koodilõiku interpreteerides saab selgitada selle mõju olekule:

\begin{align*}
  & \texttt{\footnotesize FETCH y}    & e = & [y]      &\ s & = []\\
  & \texttt{\footnotesize FETCH z}    & e = & [z,y]    &\ s & = []\\
  & \texttt{\footnotesize SUB}        & e = & [(z-y)]  &\ s & = []\\
  & \texttt{\footnotesize STORE x}    & e = & []       &\ s & = [x \mapsto (z-y)]\\
  & \texttt{\footnotesize FETCH z}    & e = & [z]      &\ s & = [x \mapsto (z-y)]\\
  & \texttt{\footnotesize FETCH x}    & e = & [x,z]    &\ s & = [x \mapsto (z-y)]\\
  & \texttt{\footnotesize MUL}        & e = & [(x*z)]  &\ s & = [x \mapsto (z-y)]\\
  & \texttt{\footnotesize STORE w}    & e = & []       &\ s & = [x \mapsto (z-y), w \mapsto (x*z)] = 
    [x \mapsto (z-y), w \mapsto ((z-y)*z)] \\
\end{align*}
\begin{align*}
  & \texttt{\footnotesize FETCH z}  & e = & [z]             &\ s & = []\\
  & \texttt{\footnotesize DUP}      & e = & [z,z]           &\ s & = []\\
  & \texttt{\footnotesize FETCH y}  & e = & [y,z,z]         &\ s & = []\\
  & \texttt{\footnotesize SWAP}     & e = & [z,y,z]         &\ s & = []\\
  & \texttt{\footnotesize SUB}      & e = & [(z-y),z]       &\ s & = []\\
  & \texttt{\footnotesize DUP}      & e = & [(z-y),(z-y),z] &\ s & = []\\
  & \texttt{\footnotesize STORE x}  & e = & [(z-y),z]       &\ s & = [x \mapsto (z-y)]\\
  & \texttt{\footnotesize MUL}      & e = & [((z-y)*z)]     &\ s & = [x \mapsto (z-y)]\\
  & \texttt{\footnotesize STORE w}  & e = & []              &\ s & = [x \mapsto (z-y), w \mapsto ((z-y)*z)]\\
\end{align*}

Mõlema koodilõigu käivitamisel saadud lõpptulemused (e ja s) on võrdsed,
mistõttu võib väita, et koodilõigud on  samatähenduslikud käivitamise
lõpptulemuse osas.
