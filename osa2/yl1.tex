\section{Ülesanne 1}

Osaliselt järjestatud hulk \( (P_{fin}(X), \subseteq) \) ei ole ccpo, sest ccpo
eelduseks on ahelatel vähima ülemise raja olemasolu, mida ei ole võimalik leida
moodustatud hulgas. Kuigi moodustatud hulga elementideks on lõplikud hulgad, on
nende hulkade hulk lõpmatu hulga X lõpmatuse tõttu.

\section{Ülesanne 2}

Ccpo olemasolu tuvastamiseks tuleb esmalt kontrollida hulga \((D_0 \times D_1,
\sqsubseteq )\) osaliselt järjestatuks olemist, kontrollides refleksiivsust,
transitiivsust ja antisümmeetrilisust hulga elementidel ning leida ahelates
vähim ülemine raja:

Osalise järjestatuse kontroll:

1.refleksiivus: \((d_0, d_1) \sqsubseteq (d_0, d_1)\) kehtib (tehtes sisaldub
võrdsus), sest \(d_0 = d_0\) ja \(d_1 \sqsubseteq_0 d_1 \)

2.transitiivsus: kui \((d_0, d_1) \sqsubseteq (e_0, e_1)\) ja \((e_0, e_1)
\sqsubseteq (f_0, f_1)\), siis kehtib ka \((d_0, d_1) \sqsubseteq (f_0, f_1)\)

3.antisümmeetrilisus: kui \((d_0, d_1) \sqsubseteq (e_0, e_1)\) ja \((e_0, e_1)
\sqsubseteq (d_0, d_1)\), siis \((d_0, d_1) = (e_0, e_1)\)

Vähima ülemise raja tuvastamine:

Kui \((D_0, \sqsubseteq_0)\) ja \((D_1, \sqsubseteq_1)\) on ccpo-d ja tehe
\(\times\) on mõlema operandi suhtes monotoonne (ehk \(f_0:D_0 \rightarrow D_0
\times D_1\) on monotoonne ja \(f_1:D_1 \rightarrow D_0 \times D_1\) on
monotoonne), siis raamatu lemma 4.30 järgi on ka \((D_0 \times D_1,
\sqsubseteq)\) ccpo ja sellel eksisteerib vähim ülemine raja.

\section{Ülesanne 3}

