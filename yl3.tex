\section{Ülesanne 3}
\subsection{Osa 1 - \(b?a_0:a_1\)}

Aritmeetikaavaldise \(b?a_0:a_1\) toetamiseks on tarvis lisada järgmised reeglid:

\(
\cfrac
  {\langle a_0,s \rangle \rightarrow _\mathrm{Aexp} z}
  {\langle b?a_0:a_1,s \rangle  \rightarrow _\mathrm{Aexp} z}
  \;
  \text{if}  B\llbracket b \rrbracket = \text{tt}
\)

ja

\(
\cfrac
  {\langle a_1,s \rangle \rightarrow _\mathrm{Aexp} z}
  {\langle b?a_0:a_1,s \rangle  \rightarrow _\mathrm{Aexp} z}
  \;
  \text{if} \; B\llbracket b \rrbracket = \text{ff}
\)

\subsection{Osa 2 - \(S \triangleright a \)}

Aritmeetikaavaldise \(S \triangleright a \) toetamiseks tuleb kõigepealt
käivitada statemnt \(S\) etteantud olekuga \(s\), mille tulemusena tekib olek
\(s'\).  Olekut \(s'\) kasutatakse aritmeetikaavaldise \(a\) rehkendamises, mis
väljastab arvu \(z\). Olek ei kandu edasi, mistõttu on \(S\) käskude mõju
globaalsele olekule mittemõjuv.

\vspace*{8pt}

\(
\cfrac
  {\langle S,s \rangle \rightarrow s', \quad \langle a, s' \rangle \rightarrow _\mathrm{Aexp} z}
  {\langle S \triangleright a,s \rangle  \rightarrow _\mathrm{Aexp} z}
\)


