\section{Ülesanne 2}

Hulga \((D_0 \times D_1, \sqsubseteq )\) ccpo-ks olemise näitamiseks on tarvis
veenduda, et eksisteerib \(\sqcup Y \) üle kõikide moodustatud hulga ahelate Y.

Selleks tuleb vaadata eraldi kahte hulga elemente defineeriva reegli osa:

1. Reegli \( (d_0 \neq e_0) \wedge (d_0 \sqsubseteq e_0) \) moodustatud elemendid
on moodustatud sarnaselt \((D_0, \sqsubseteq_0)\) moodustamisele ning omavad
\(\sqcup Y \) seetõttu ka moodustatud hulgas \((D_0 \times D_1, \sqsubseteq )\).

2. Reegli \( (d_0 = e_0) \wedge (d_1 \sqsubseteq e_1) \) moodustatud elemendid
on moodustatud sarnaselt \((D_1, \sqsubseteq_1)\) moodustamisele ning omavad
\(\sqcup Y \) seetõttu ka moodustatud hulgas \((D_0 \times D_1, \sqsubseteq )\).

Kuna \((D_0 \times D_1, \sqsubseteq )\) on moodustatud karteesia korrutisena,
eksisteerib ka element \( (\sqcup d_0, \sqcup d_1 ) \), mis on moodustatud hulga
\(\sqcup Y \).
