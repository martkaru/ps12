\section{Ülesanne 2}

\subsection{a}

Kui b_1 = false \; ja \; b_2 = false \; ning \; b_1 = false \; ja \; b_2
= true:

\[
  \frac{
    \langle S_2, s \rangle \rightarrow s'
  }
  {
    \langle \text{if } (b_1 \wedge b_2) \text{ then } S_1 \text{ else }
    S_2, s \rangle \rightarrow s'
  }
\]

Millest annab tuletada:

\[
  \frac{
    \langle S_2, s \rangle \rightarrow s'
  }{
    \langle \text{if } b_1 \text{ then } (\text{if } b_2 \text{ then } S_1 \text{ else }
    S_2)) \text{ else } S_2, s \rangle \rightarrow s'
  }
\]

Teistpidi:

\[
  \frac{
    \langle S_2, s \rangle \rightarrow s'
  }{
    \langle \text{if } b_1 \text{ then } (\text{if } b_2 \text{ then } S_1 \text{ else }
    S_2)) \text{ else } S_2, s \rangle \rightarrow s'
  }
\]

Millest annab samuti tuletada:

\[
  \frac{
    \langle S_2, s \rangle \rightarrow s'
  }
  {
    \langle \text{if } (b_1 \wedge b_2) \text{ then } S_1 \text{ else }
    S_2, s \rangle \rightarrow s'
  }
\]


Kui b_1 = true \; ja \; b_2 = false:

\[
  \frac{
    \langle S_2, s \rangle \rightarrow s'
  }
  {
    \langle \text{if } (b_1 \wedge b_2) \text{ then } S_1 \text{ else }
    S_2, s \rangle \rightarrow s'
  }
\]

Millest annab tuletada:

\[
  \frac{
    \langle S_2, s \rangle \rightarrow s'
  }{
    \cfrac{
      \langle \text{if } b_2 \text{ then } S_1 \text{ else }
      S_2), s \rangle \rightarrow s'
    }{
      \langle \text{if } b_1 \text{ then } (\text{if } b_2 \text{ then } S_1 \text{ else }
      S_2)) \text{ else } S_2, s \rangle \rightarrow s'
    }
  }
\]

Teistpidi:

\[
  \frac{
    \langle S_2, s \rangle \rightarrow s'
  }{
    \cfrac{
      \langle \text{if } b_2 \text{ then } S_1 \text{ else }
      S_2), s \rangle \rightarrow s'
    }{
      \langle \text{if } b_1 \text{ then } (\text{if } b_2 \text{ then } S_1 \text{ else }
      S_2)) \text{ else } S_2, s \rangle \rightarrow s'
    }
  }
\]

Millest annab samuti tuletada:

\[
  \frac{
    \langle S_2, s \rangle \rightarrow s'
  }
  {
    \langle \text{if } (b_1 \wedge b_2) \text{ then } S_1 \text{ else }
    S_2, s \rangle \rightarrow s'
  }
\]


Kui b_1 = true \; ja \; b_2 = true:

\[
  \frac{
    \langle S_1, s \rangle \rightarrow s'
  }
  {
    \langle \text{if } (b_1 \wedge b_2) \text{ then } S_1 \text{ else }
    S_2, s \rangle \rightarrow s'
  }
\]

Millest annab tuletada:

\[
  \frac{
    \langle S_1, s \rangle \rightarrow s'
  }{
    \cfrac{
      \langle \text{if } b_2 \text{ then } S_1 \text{ else }
      S_2), s \rangle \rightarrow s'
    }{
      \langle \text{if } b_1 \text{ then } (\text{if } b_2 \text{ then } S_1 \text{ else }
      S_2)) \text{ else } S_2, s \rangle \rightarrow s'
    }
  }
\]

Teistpidi:

\[
  \frac{
    \langle S_1, s \rangle \rightarrow s'
  }{
    \cfrac{
      \langle \text{if } b_2 \text{ then } S_1 \text{ else }
      S_2), s \rangle \rightarrow s'
    }{
      \langle \text{if } b_1 \text{ then } (\text{if } b_2 \text{ then } S_1 \text{ else }
      S_2)) \text{ else } S_2, s \rangle \rightarrow s'
    }
  }
\]

Millest annab samuti tuletada:

\[
  \frac{
    \langle S_1, s \rangle \rightarrow s'
  }
  {
    \langle \text{if } (b_1 \wedge b_2) \text{ then } S_1 \text{ else }
    S_2, s \rangle \rightarrow s'
  }
\]


