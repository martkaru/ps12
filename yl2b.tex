\subsection{b}

\subsubsection{ Kui \(b_1 = false\) ja \(b_2 = false\) ning \(b_1 = false\) ja \(b_2 = true\), siis}

\[
    \langle
    \text{while } b_1 \text{ do } (\text{while } (b_1 \wedge b_2) \text{ do } S), s
    \rangle \rightarrow s
\]

While-tsükli keha ei täideta ning olek s ei muutu. Sama juhtub ka teise avaldisega:

\[
    \langle
    \text{while } b_1 \text{ do } (\text{if } b_2 \text{ then } S \text{ else }
    (\text{while true do skip})) , s
    \rangle \rightarrow s
\]


\subsubsection{Kui \(b_1 = true\) ja \(b_2 = false\)}

\[
  \frac{
    T_1
    \quad
    T_2
  }{
    \langle
    \text{while } b_1 \text{ do } (\text{while } (b_1 \wedge b_2) \text{ do } S), s
    \rangle \rightarrow s 
  }
\] 
\[T_1 = \langle \text{while } (b_1 \wedge b_2) \text{ do } S, s \rangle \rightarrow s\]
\[T_2 = \langle \text{while } b_1 \text{ do } (\text{while } (b_1 \wedge b_2)), s \rangle \rightarrow s\]

Kuna \(T_1\) tsükli keha ei läbita, jääb olek s muutmata ning \(T_1\) saab võrdsustada
oma lõpptulemuse osas käsuga skip. Samuti võib sisuliselt \(T_2\) asendada lausega:

\[T_2 = \langle \text{while } b_1 \text{ do } \text{skip}), s \rangle \rightarrow s\]

Kuna \(b_1 = true\), siis saab sellest ehitada teise avaldise tuletuspuu:


\[
  \frac{
    \cfrac{
      \langle \text{while true do skip}, s \rangle
    }{
      \langle \text{if } b2 \text{ then } S \text{ else } (\text{while true do
      skip}))
      ,s
      \rangle
    }
    \quad
    T_3
  }{
    \langle
    \text{while } b1 \text{ do } (\text{if } b2 \text{ then } S \text{ else }
    (\text{while true do skip}))
    ,s
    \rangle \rightarrow s 
  }
\] 

\[T_3 = 
    \langle
    \text{while } b1 \text{ do } (\text{if } b2 \text{ then } S \text{ else }
    (\text{while true do skip}))
    ,s
    \rangle \rightarrow s 
  \]

\subsubsection{Kui \(b_1 = true\) ja \(b_2 = true\)}

Esimene avaldis:

\[
  \frac{
    \cfrac{
      \cfrac{
        T_1
      }{
        \langle
        S, s
        \rangle \rightarrow s_1
      }
      \quad
      \cfrac{
        T_2
      }{
        \langle
        \text{while } (b_1 \wedge b_2) \text{ do } S, s_1
        \rangle \rightarrow s_2
      }
    }{
      \langle
      \text{while } (b_1 \wedge b_2) \text{ do } S, s
      \rangle \rightarrow s_2
    }
    \quad
    \langle
    \text{while } b_1 \text{ do } (\text{while } (b_1 \wedge b_2) \text{ do } S), s_2
    \rangle \rightarrow s'
  }{
    \langle
    \text{while } b_1 \text{ do } (\text{while } (b_1 \wedge b_2) \text{ do } S), s
    \rangle \rightarrow s'
  }
\] 

Kasutades tuletuspuid \(T_1\) ja \(T_2\) ja eeldades, et \((b_1 \wedge b_2)\)
asemel saab kasutada while tingimuseks ainult \(b_1\) saab ehitada teise avaldise:

\[
\frac{
  \frac{
    \cfrac{
      T_1
    }{
      \langle
      S, s
      \rangle \rightarrow s_1
    }
  }{
    \langle
    \text{if } b2 \text{ then } S \text{ else }
    (\text{while true do skip})
    ,s
    \rangle \rightarrow s_1
  }
  \quad
  \frac{
    \cfrac{
      T_2
    }{
      \langle
      \text{while } b_1 \text{ do } S, s_1
      \rangle \rightarrow s'
    }
  }{
    \langle
    \text{while } b1 \text{ do } (\text{if } b2 \text{ then } S \text{ else }
    (\text{while true do skip}))
    ,s_1
    \rangle \rightarrow s'
  }
}{
  \langle
  \text{while } b1 \text{ do } (\text{if } b2 \text{ then } S \text{ else }
  (\text{while true do skip}))
  ,s
  \rangle \rightarrow s'
}
\]
